%% Generated by Sphinx.
\def\sphinxdocclass{report}
\documentclass[letterpaper,10pt,english]{sphinxmanual}
\ifdefined\pdfpxdimen
   \let\sphinxpxdimen\pdfpxdimen\else\newdimen\sphinxpxdimen
\fi \sphinxpxdimen=.75bp\relax
\ifdefined\pdfimageresolution
    \pdfimageresolution= \numexpr \dimexpr1in\relax/\sphinxpxdimen\relax
\fi
%% let collapsible pdf bookmarks panel have high depth per default
\PassOptionsToPackage{bookmarksdepth=5}{hyperref}

\PassOptionsToPackage{warn}{textcomp}
\usepackage[utf8]{inputenc}
\ifdefined\DeclareUnicodeCharacter
% support both utf8 and utf8x syntaxes
  \ifdefined\DeclareUnicodeCharacterAsOptional
    \def\sphinxDUC#1{\DeclareUnicodeCharacter{"#1}}
  \else
    \let\sphinxDUC\DeclareUnicodeCharacter
  \fi
  \sphinxDUC{00A0}{\nobreakspace}
  \sphinxDUC{2500}{\sphinxunichar{2500}}
  \sphinxDUC{2502}{\sphinxunichar{2502}}
  \sphinxDUC{2514}{\sphinxunichar{2514}}
  \sphinxDUC{251C}{\sphinxunichar{251C}}
  \sphinxDUC{2572}{\textbackslash}
\fi
\usepackage{cmap}
\usepackage[T1]{fontenc}
\usepackage{amsmath,amssymb,amstext}
\usepackage{babel}



\usepackage{tgtermes}
\usepackage{tgheros}
\renewcommand{\ttdefault}{txtt}



\usepackage[Bjarne]{fncychap}
\usepackage{sphinx}

\fvset{fontsize=auto}
\usepackage{geometry}


% Include hyperref last.
\usepackage{hyperref}
% Fix anchor placement for figures with captions.
\usepackage{hypcap}% it must be loaded after hyperref.
% Set up styles of URL: it should be placed after hyperref.
\urlstyle{same}

\addto\captionsenglish{\renewcommand{\contentsname}{User Guide}}

\usepackage{sphinxmessages}
\setcounter{tocdepth}{2}



\title{PDS Example Namespace}
\date{Apr 25, 2022}
\release{}
\author{NASA Planetary Data System}
\newcommand{\sphinxlogo}{\vbox{}}
\renewcommand{\releasename}{}
\makeindex
\begin{document}

\pagestyle{empty}
\sphinxmaketitle
\pagestyle{plain}
\sphinxtableofcontents
\pagestyle{normal}
\phantomsection\label{\detokenize{index::doc}}


\sphinxAtStartPar
This where you would put the description of the namespace

\sphinxstepscope

\sphinxAtStartPar
\{date\}\\
\{author\}
\begin{quote}

\sphinxAtStartPar
Note to authors who use this outline: The outline is a
suggestion only. It includes the minimum of content needed to inform the
dictionary user. Authors are expected to tailor the outline to their particular
purposes, elaborating and providing context as needed.
\end{quote}


\chapter{Introduction}
\label{\detokenize{user/user-guide:introduction}}\label{\detokenize{user/user-guide::doc}}\begin{enumerate}
\sphinxsetlistlabels{\arabic}{enumi}{enumii}{}{.}%
\item {} 
\sphinxAtStartPar
Purpose of this User’s Guide

\item {} 
\sphinxAtStartPar
Audience

\item {} 
\sphinxAtStartPar
Applicable Documents

\end{enumerate}


\chapter{Overview of the \{name\} Local Data Dictionary}
\label{\detokenize{user/user-guide:overview-of-the-name-local-data-dictionary}}
\sphinxAtStartPar
\sphinxstyleemphasis{What is this dictionary for? What kinds of products might
use this dictionary? Who is the steward of this dictionary (person and node
name)? How often is it updated? To whom should questions about it be directed?
(Give an email address or link to a page with contact information.)}


\chapter{How to Include the \{name\} Local Data Dictionary in a PDS4 Label}
\label{\detokenize{user/user-guide:how-to-include-the-name-local-data-dictionary-in-a-pds4-label}}
\sphinxAtStartPar
\sphinxstyleemphasis{Briefly explain the form that a discipline dictionary
takes: the input ingest file, the output schema, Schematron, and other files,
which ones are necessary and which ones may be ignored.}

\sphinxAtStartPar
\sphinxstyleemphasis{Give the URL at PDS where the dictionary files may be downloaded.}

\sphinxAtStartPar
\sphinxstyleemphasis{Give (and explain) a label snippet showing the beginning lines of a
label to demonstrate the use of the dictionary schema and schematron
in the xml prolog and the root tag (preferably Product\_Observational).}

\sphinxAtStartPar
\sphinxstyleemphasis{Give (and explain) a label snippet showing the location
of the classes and attributes in the label; i.e., inside the Discipline\_Area
tag. Collapse the lower level classes if needed for brevity.}

\sphinxAtStartPar
\sphinxstyleemphasis{In general, for label snippets use a fixed\sphinxhyphen{}width font and
consistent indentation. Color\sphinxhyphen{}coding is helpful. Label snippets copied from an
Oxygen editor window will retain their color\sphinxhyphen{}coding when pasted here.}


\chapter{Organization of Classes and Attributes}
\label{\detokenize{user/user-guide:organization-of-classes-and-attributes}}
\sphinxAtStartPar
\sphinxstyleemphasis{Give a schematic diagram or a list showing the hierarchy of
classes in order of appearance in label. Refer the reader to the Definitions
section for complete definitions. An example of such a list is given on the
page \sphinxhref{http://sbndev.astro.umd.edu/wiki/Filling\_Out\_the\_Spectral\_Dictionary\_Classes\#.3CCircular\_FOV.3E}{Filling Out The Spectral Dictionary Classes}
on the PDS Small Bodies Node wiki. In this example the names of classes and attributes have hyperlinks to
their definitions further down the page, a useful lookup tool.}

\sphinxAtStartPar
\sphinxstyleemphasis{The author should take into consideration the complexity
of the dictionary when organizing this section.  If the hierarchy is large or
complicated, it may be helpful to break it down by class as shown in the
following subsections, but don’t forget to provide a high\sphinxhyphen{}level view of how the
classes relate to one another.}


\section{Class 1}
\label{\detokenize{user/user-guide:class-1}}
\sphinxAtStartPar
\sphinxstyleemphasis{What is this class for?}

\sphinxAtStartPar
\sphinxstyleemphasis{Give a schematic diagram or a list of the attributes in this class in order of
appearance in label. Refer reader to Definitions section for complete definitions.}

\sphinxAtStartPar
\sphinxstyleemphasis{Give label snippets showing use of the class and attributes, with annotations
as appropriate. Refer reader to Examples section for complete examples.}

\sphinxAtStartPar
\sphinxstyleemphasis{Explain why some things are required and others are optional.}

\sphinxAtStartPar
\sphinxstyleemphasis{List and explain any rules that apply to this class (e.g. from Schematron).}


\section{Class 2}
\label{\detokenize{user/user-guide:class-2}}
\sphinxAtStartPar
{[}repeat this subsection for each class{]}


\chapter{Definitions}
\label{\detokenize{user/user-guide:definitions}}
\sphinxAtStartPar
\sphinxstyleemphasis{Give an alphabetical list of all classes and attributes
with complete definitions. (Useful ones, not silly ones like “The
map\_projection\_name attribute provides the name of the map projection.”)}

\sphinxAtStartPar
\sphinxstyleemphasis{Include:}
\begin{itemize}
\item {} 
\sphinxAtStartPar
\sphinxstyleemphasis{Class or attribute name (indicate which it is; capitalize class names according to PDS4 standard)}

\item {} 
\sphinxAtStartPar
\sphinxstyleemphasis{PDS4 data type (ASCII\_Short\_String\_Collapsed, ASCII\_Real, ASCII\_Date, etc.)}

\item {} 
\sphinxAtStartPar
\sphinxstyleemphasis{Definition in complete sentences}

\item {} 
\sphinxAtStartPar
\sphinxstyleemphasis{Cardinality (minimum and maximum number of values permitted)}

\item {} 
\sphinxAtStartPar
\sphinxstyleemphasis{Nillable, yes or no? Explain when it is appropriate to use a nil value}

\item {} 
\sphinxAtStartPar
\sphinxstyleemphasis{Minimum and maximum numeric values, if applicable}

\item {} 
\sphinxAtStartPar
\sphinxstyleemphasis{Minimum and maximum number of characters, if applicable}

\item {} 
\sphinxAtStartPar
\sphinxstyleemphasis{List of valid values, if applicable.}

\end{itemize}


\chapter{Examples}
\label{\detokenize{user/user-guide:examples}}
\sphinxAtStartPar
\sphinxstyleemphasis{Give one or more examples of label snippets for real products, annotated as appropriate.
Make sure the examples can be successfully validated using the latest version of the PDS4
core dictionary and, of course, the dictionary described in this document.}

\sphinxstepscope


\chapter{\textless{}My\_First\_LDD\textgreater{}}
\label{\detokenize{detailed/example:my-first-ldd}}\label{\detokenize{detailed/example::doc}}
\sphinxAtStartPar
\sphinxstylestrong{REQUIRED}

\sphinxAtStartPar
Submitter: Jane Doe

\sphinxAtStartPar
My first class


\section{\textless{}my\_first\_attribute\textgreater{}}
\label{\detokenize{detailed/example:my-first-attribute}}
\sphinxAtStartPar
\sphinxstylestrong{REQUIRED}

\sphinxAtStartPar
Submitter: Jane Doe

\sphinxAtStartPar
My first attribute



\renewcommand{\indexname}{Index}
\printindex
\end{document}